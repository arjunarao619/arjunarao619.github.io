
\documentclass[12pt]{article}

\usepackage[
  a4paper,
  margin=1in,
  headsep=10pt, % separation between header rule and text
]{geometry}
\usepackage{xcolor}
\usepackage{datetime}

\newdateformat{monthyeardate}{%
  \monthname[\THEMONTH], \THEYEAR}
\usepackage{lastpage}
\usepackage{fancyhdr}
\pagestyle{fancy}
\fancyhf{} 
\lhead{\footnotesize Arjun Rao}% \lhead puts text in the top left corner. \footnotesize sets our font to a smaller size.
\rhead{\footnotesize Research Statement - \monthyeardate\today } 

\usepackage{graphicx}

\newcommand{\soptitle}{Arjun Ashok Rao}

\newcommand{\yourname}{firstname lastname}
\newcommand{\youremail}{arjunrao@link.cuhk.edu.hk}
\newcommand{\yourweb}{https://arjunashokrao.me}
\newcommand{\citeColored}[2]{{\hypersetup{citecolor=#1}\cite{‌​#2}}}
\newcommand{\HRule}[1][\medskipamount]{\par
  \vspace*{\dimexpr-\parskip-\baselineskip+#1}
  \noindent\rule{\linewidth}{0.3mm}\par
  \vspace*{\dimexpr-\parskip-.3\baselineskip+#1}}

\newcommand{\statement}[1]{\par\medskip
  \textcolor{blue}{\textbf{#1:}}\space
}
\usepackage{hyperref}
\hypersetup{
    colorlinks=True,
    linkcolor={blue},
    citecolor=blue,
    filecolor=red,      
    urlcolor=blue,
}

\begin{document}
\thispagestyle{empty}
\Large \textbf{Arjun Rao} \hfill \Large  \textbf{\textcolor{black}{\Large{Letter of Motivation} }} \\ \textcolor{gray}{\textbf{\hspace*{0.25in}\normalsize{The Chinese University of Hong Kong}}}
\HRule

\small Homepage: \href{https://arjunashokrao.me}{arjunashokrao.me} \quad Email: \href{mailto:arjunrao@link.cuhk.edu.hk}{arjunrao@link.cuhk.edu.hk}

\bigskip

\textbf{Abstract: }My Research Interests include robustness and generalization in machine learning. More particularly, I am interested in understanding adversarial machine learning from a \emph{robustness} perspective, and how adversarial robustness can shed light on theoretical concepts such as \emph{generalization} as well as \emph{representation learning} and \emph{transfer learning} \cite{Andriushchenko:273123}. My long term research goal is to understand the generalization guarantee in over-parameterized deep nets, and to develop deep learning models which are robust to adversarial and poisoning attacks in the real world. Recent work by TML in robustness \cite{croce2020robustbench,Andriushchenko:278914,Andriushchenko:273123} have helped me develop and understand ideas in my own research. I believe my experience in adversarial machine learning with a focus on robustness make me a good fit for TML Laboratory's summer research program.  \\



%Adversarial Examples, which can better be visualized as imperceptible `distributional' shifts in test-datasets are a natural consequence of the dimensionality gap between inputs and linear models on which high-dimensional inputs are trained on. Adversarial examples which lie on a low-dimensional data manifold are chief contributors to generalization error \cite{43405,stutz2019disentangling}.
%They generalize across different architectures, and can be used in a `black-box' fashion to threaten real-world deep learning models. Recent work has demonstrated the almost ubiquitous prevalence of adversarial examples in different applications of deep neural networks - encompassing classification (Image and voice classification, Facial Recognition), regression (Object detection, multi-object tracking), and reinforcement learning. The most common strategy to defend against test-time attacks has been to train models on adversarial data, thus ensuring some `robustness` against standard attacks. Interestingly, these adversarially trained (robust) models exhibit intriguing properties not seen in standard deep nets. My current research aims on understanding and leveraging these properties to develop deep nets that are interpretable, secure, and accurate on unseen data. 
   % \\
\textbf{Current Research:} At The Chinese University of Hong Kong, my research primarily involves developing application-specific adversarial defenses. Stereo-Vision, most commonly used in autonomous driving applications were shown to be vulnerable to adversarial attacks that primarily distort the \emph{disparity} perception in the rectified stereo pair (Figure \ref{1}). My research developed a completely cyber-physical approach to conduct adversarial training with left-right feature map regularization while ensuring local linearity of the loss surface \cite{me}. I also worked on more exploratory research on understanding distributed deep learning and how over-parameterized neural networks can generalize in a distributed setting. \\

\textbf{Interest in Theory of Machine Learning Laboratory: } TML's most recent work on benchmarking models in adversarial robustness is a real step towards scoring the somewhat \emph{chaotic} progress in this area as researchers often have strong incentives to evaluate on weak attack methods in favor of high robust accuracy. At TML, I have been following Dr. Andriuschenko's work on adversarial examples since my undergraduate sophomore year. Dr. Andriuschenko's work on fast adversarial training with FGSM along with a new regularization method was eye-opening, especially for those who believe that PGD attacks can only be fended off with high-cost, multi-step PGD adversarial training on different metrics ($l_{1},l_{2},l_{\infty}$). TML's work on transferability of adversarial examples has also been very exciting \cite{Croce:278915,Andriushchenko:273123}. Black-box adversarial examples are a better indication of real-world attacks on vision and language systems, and also helps explain the phenomenal generalization ability of these attacks. An important direction of my research would be to address and explain this phenomenon. \\

\textbf{Future Research milestones: }The disproportionate progress in empirical studies opposed to theoretical explanations in machine learning have to be bridged in the near future. Broadly, I believe that a large part of my future research would be to understand and evaluate training algorithms for deep neural networks. Secondly, I aim to leverage the property of robustness to help develop deep learning models that are interpretable and explainable. Through my research, I hope to see through these brittle and unexplainable black-boxes we call neural networks. Theory of Machine Learning Laboratory at EPFL's work on bridging this theory-result gap, especially in areas of interest such as robustness and generalization have been incredibly exciting for young researchers. My future research goal is to preferably pursue a joint master's + PhD in either computer science, or computational mathematics. EPFL and TML's reputation have demonstrated that living and working in Switzerland can help me make great progress in my academic research goals. 

%\textbf{Robustness and Generalization:} Understanding adversarial robustness may help develop a better understanding about broad theoretical questions such as those on local minima, generalization in over-parameterized networks, or the reasoning behind flat vs sharp global optimums. The consensus regarding loss landscape geometries have been that flatter minima basins lead to better generalization. This hypothesis is a potential explanation to the robustness-accuracy tradeoff in robust models \cite{tsipras2018robustness} since robust models qualitatively exhibit sharper minima with lesser basin volume. 
%This generalization error occurs when the adversarial sample lies within the boundary of the low-dimensional data manifold \cite{zeng2019adversarial}. 
%However, the lack of generalization in robust models is not absolute, and a large number of studies which study this generalization gap are only empirical. Explicit regularization like enforcing local linearity of the loss surface combats the sharp minima problem in robust models. It is also possible that the good generalization in robust models can be due special cases of sharp minima generalization \cite{10.5555/3305381.3305487}.\\

%\textbf{Robustness and Representations:} My research also studies how robustness leads to interpretable machine learning and why robust networks contain representations that align more with the human assumptions of intermediate neural network features. For example, previous empirical studies have shown that models trained against adversarial examples exhibit more human-perceptible input-gradient visualizations, as well as a smooth interpolation between image classes. Robust representations provide a high-level embedding of the input such that similar-looking classes have intermediate representations that are semantically similar \cite{engstrom2019adversarial}. In fact, performing adversarial training on very small perturbation sets might not even lead to robustness and security, but can still lead to better representations. While the intuition that neural networks truly learn interpretable representations is often assumed to be true, it is not certain. Answering questions on why robustness leads to better representations would equal taking a large step towards demystifying these `black-boxes'. \\

%\textbf{Robustness and Transferability:} The inherent gap in dimensionality between inputs and linear models on which those high-dimensional inputs are trained on is a major explanation for the existence of adversarial examples. In fact, studies claim that adversarial examples which lie on a low-dimensional data manifold are chief contributors to generalization error \cite{43405,stutz2019disentangling}. Since adversarial examples generalize across different architectures, they are transferable and can be used in a `black-box' fashion to threaten real-world deep learning models. However, robustness might also transfer through input gradients \cite{chan2020thinks}


\begin{figure}
	   \center{\includegraphics[scale=0.4]
	       {fig2}}
	  \caption{\label{1} \footnotesize{(My recent research on Stereo-Robustness) Visualization of 3-D bounding box and object proposals for a Stereo-Vision model before (Row 1) and after (Row 2) a small $\epsilon$ projected gradient descent adversarial attack. Adversarial examples cause a large number of false-positive, high-confidence region proposals. Stereo images cause a disparity perception and orientation estimation error which predicts incorrect bounding boxes for the autonomous system.}}
	\end{figure}

%\begin{figure}
%	   \center{\includegraphics[scale=1]
%	       {fig}}
%	  \caption{\label{1} \footnotesize{"Side-Effects" of adversarial robustness include sharper optimum minima from the image space (Row 1, and smoother interpolation between target classes}}
%	\end{figure}

%\textbf{Future Research Goals:} Machine Learning has demonstrated incredible progress over the past decade. However, the disproportionate progress in empirical studies opposed to theoretical explanations have to be bridged in the near future. Broadly, I believe that a large part of my research is to understand and evaluate training algorithms for deep neural networks. An integral part of my future research goal is to see through the opaqueness of over-parameterized deep nets. More specifically, I am interested in studying the generalization guarantee that deep neural nets offer on modern datasets. It is understood that SGD - a variation of gradient descent that updates network weights based on a `mini-batch' of input-data performs a certain `implicit' regularization that is responsible for their excellent generalization results. However, we know very little about optimization trajectories that these powerful algorithms take, especially for over-parameterized nets with high-dimensional, highly non-convex loss landscapes. My near-term research goal is to demonstrate and visualize these phenomenon to provide better intuitions for other researchers. Secondly, I am interested in safe and robust machine learning that can be used at scale. My previous work on studying adversarial attacks on autonomous systems is an example of an effort in this interesting and needed research direction. 
%My broad research goal is to develop an understanding of deep neural networks that can potentially explain the presence of these small and imperceptible adversarial examples. This understanding should also help other explain other unexplained dichotomies and findings in adversarial machine learning - like how robust training leads to better representations but worse generalization with sharper global minima in their loss landscapes. 
\bibliographystyle{IEEEtran}
\bibliography{ref.bib}
\end{document}